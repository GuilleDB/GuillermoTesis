Inteligencia Artificial: Según Lasse Rouhiainen, la Inteligencia artificial es un conjunto de algoritmos que tienen la capacidad de tomar decisiones cómo lo haría un ser humano, aplicable a la mayoría de situaciones, como la detección y clasificación de un objeto, aumentar el desempeño de algunos procesos o la detección de riesgos (Rouhiainen, 2018).

Aprendizaje Automático: Es una rama importante de la Inteligencia Artificial que es aplicada a cualquier disciplina, utilizando conocimientos de informática, estadísticas e ingeniería para la predicción o evaluación de los objetos después de haber sido entrenado con datos históricos. (Tao, Q. e.at., 2021).

Aprendizaje Profundo: Subconjunto del Aprendizaje automático, con la capacidad de reconocer patrones con un gran volumen de datos, estructurados y no estructurados, entrenados por redes neuronales artificiales que intentan emular el cerebro humano y que, a comparación del aprendizaje automático, no requiere de la intervención humana para el modelo (IBM, s.f.).

Aprendizaje por Refuerzo Profundo (DRL): Técnica que combina el aprendizaje profundo con aprendizaje por refuerzo. Se utiliza una red neuronal de aprendizaje profundo que aprende por cada acción en tiempo real se presente, un ejemplo de esto es en un juego de Ajedrez (gamco, s.f).

Graph Neural Networks: Técnica que utiliza la potencia de predicción del aprendizaje profundo para predecir objetos(nodos) que están relacionados (bordes). Tiene la capacidad de predecir a nivel de nodo, edges o gráficos (Merritt, 2022).

Red convolucional-LSTM basada en grafos (GCLSTM): Es una red neuronal que integra LSTM para la predicción dinámica de enlaces de red.
Regresión LOESS: Técnica no paramétrica para eliminar los efectos de ruido de los datos para suavizar una curva.

Optimización de Política Proximal (PPO): Es un método que impide al modelo realizar modificaciones adicionales que causen inestabilidad, además de ser escalable y emplea un enfoque basado en regiones de confianza y eficiencia del muestreo (Greyrat, 2022).

