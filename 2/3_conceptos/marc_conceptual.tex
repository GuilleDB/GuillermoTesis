\subsection{Aprendizaje Profundo}

Subconjunto del Aprendizaje automático, con la capacidad de reconocer patrones con un gran volumen de datos, estructurados y no estructurados, entrenados por redes neuronales artificiales que intentan emular el cerebro humano y que, a comparación del aprendizaje automático, no requiere de la intervención humana para el modelo \parencite{gl_IBMIA}.

\subsection{Aprendizaje Profundo (RL)}
Subconjunto del Aprendizaje Automático, su objetivo es determinar qué acciones debe un agente del software elegir para maximizar una función de recompensa mediante la interacción dinámica del entorno. Su principal ventaja frente a otros algoritmos de Aprendizaje Automático, es que permite que el modelo aprenda sin necesidad de que intervenga el programador \parencite{gl_techopedia}.

\subsection{Aprendizaje Profundo por Refuerzo (DRL)}

Técnica que combina el aprendizaje profundo con aprendizaje por refuerzo. Se utiliza una red neuronal de aprendizaje profundo que aprende por cada acción en tiempo real se presente, un ejemplo de esto es en un juego de Ajedrez \parencite{gl_Gamco}.

\subsection{Redes Neuronales Profundas (DNN)}

Las Redes Neuronales Profundas, o Deep Neural Networks (DNN) en inglés, es una de las categorías de modelos del Aprendizaje Profundo, esta red está compuesta por varias capas de nodos que tienen como objetivo simular las neuronas del cerebro humano, su estructura está conformada por capas de entrada, de red y de salida, cada capa realiza cálculos de aprendizaje \parencite{pr_talaei}. 

\subsection{Graph Neural Network (GNN)}

Técnica que utiliza la potencia de predicción del aprendizaje profundo para predecir objetos(nodos) que están relacionados (bordes). Tiene la capacidad de predecir a nivel de nodo, edges o gráficos\parencite{gl_Nvidia}. 

\subsection{Zona Urbana}

Lugar compuesto por ciudades, con alto nivel de población, además de estar compuesta por calles, avenidas, edificación y comercios. Además de que se encuentra la mayoría de servicios básicos, como también empresas y espacios burocráticos del Estado, las zonas urbanas se caracterizan por tener una gran variedad de autopistas, vías de ferrocarril, estaciones y puertos\parencite{gl_humanidades}.

\subsection{Delincuencia}

Es la acción de cometer actos ilegales que van en contra de las leyes impuestas por el estado, son cometidas a nivel individual o por grupos delictivos. Existen diversos tipos de delincuencias, entre ellas, el robo, violencia doméstica, suplantación de identidad, asalto, entre otros\parencite{gl_concepto}.


\subsection{Ruta}

Trayecto el cual una persona puede transitar desde un lugar a otro, el cuál puede variar según el propósito de la persona. También una ruta puede definirse como un camino, vía o carretera de diferentes lugares geográficos\parencite{gl_ruta}.

\subsection{Riesgo}

Es la probabilidad de que ocurra un evento y sus efectos negativos frente a una persona u objeto, como el daño o la situación de peligro. El riesgo puede ser representado mediante una medida probabilística con el fin de saber qué evento puede amenazar con el cumplimiento de un objetivo de una organización o, en otros contextos, para saber el nivel de seguridad ciudadana\parencite{gl_riesgo}.